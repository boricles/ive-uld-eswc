So far, Linked Data principles and practices are being adopted by an increasing number of data providers, getting as result a global data space on the Web containing hundreds of LOD datasets \cite{Heath_Bizer_2011}. In this context it is important to promote the reuse and linkage of datasets, and to this end, it is necessary to know the structure of datasets. One step forward for knowing in depth the structure of a given dataset is to provide a summary of the dataset.

In this paper we present fancy-name\footnote{\footnotesize \url{http://homepages.abdn.ac.uk/honghan.wu/pages/summ/}}, a simple tool to help users get a quick understanding of RDF dataset ...

The rest of paper is organized as follows: firstly, we introduce the details of the summarisation definition and generation. Then, we present the good properties of this summarisation and propose a set of useful services from the summarisation, both of which can be utilised to guide data exploitation tasks like query writing, ontology reasoning, data compression, and data diagnosis. Finally, we evaluate our approach in several typical data exploitation scenarios. Experiments on real word datasets show that our approach can guide very efficient data exploitation.

