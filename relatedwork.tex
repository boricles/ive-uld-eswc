There are existing work such as (1) \emph{LODStats}\footnote{\footnotesize \url{http://stats.lod2.eu/}} that provides the information related to a dataset, and (2) \emph{make-void} \footnote{\footnotesize \url{https://github.com/cygri/make-void}} that computes statistics about RDF files. However, LODStats is thought for the whole set of LOD datasets registered in The Data Hub \footnote{\footnotesize \url{http://thedatahub.com}}, and it is based on declarative descriptions of those datasets; and \emph{make-void} is thought for RDF files but not for RDF datasets.

Moreover, there are some existing efforts such as Zhang et al.\cite{ZhangCQ07} for summarising ontologies based on RDF sentence graphs, and Li et al. \cite{LiM10} for user-driven ontology summarisation. However, both help the understanding rather than the exploitation, which is usually task oriented.

