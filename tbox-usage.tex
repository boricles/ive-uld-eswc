\vspace{-2mm}
Given an RDF graph, the summarisation is to generate a condensed description which can facilitate data exploitations. The main novelty of our summarisation approach is that it summarises an RDF graph by another much smaller graph structure based on atomic graph patterns. The linking structure in such summary graph can be utilised to significantly decrease search spaces in various data exploitation tasks e.g., query generation and query answering. Furthermore, statistic results of pattern instances are precomputed and attached to the summary, which can help both better understanding about the dataset and more efficient exploitation operations on it. 

In an RDF graph, we call its non-literal nodes as entities. For an entity $e$ in an RDF graph $G$, we can get a data block for it by extracting triples in $G$ each of which has $e$ as its subject or object. We call such kind of data blocks as Entity Description Blocks (EDBs for short). 

\noindent \textbf{Entity Description Pattern} For an EDB, it can be summarised by a notion of entity description pattern (Definition~\ref{def:edp}). EDP, the short name for entity description pattern, is the atomic graph pattern in our summarisation model. 

\begin{definition}
\label{def:edp} 
(Entity Description Pattern) Given an entity description block $B_e$, its description pattern is a tuple $P_e=(C_e,A_e,R_e,V_e)$, where
\begin{itemize}
\item $C_e=\{c_i |<e, rdf:type,c_i> \in G\}$   is called as the class component; 
\item $A_e=\{p_i |<e,p_i,l_i> \in G \text{ and $l_i$  is a literal}\}$  is called as the attribute component;
\item $R_e=\{r_i |<e,r_i,o_i> \in G \text{ and $o_i$  is a URI resource or blank node}\}$  is called as the relation component;
\item $V_e=\{v_i |<s_i,v_i,e> \in G\}$ is called as the reverse relation component.
\end{itemize}
\end{definition}

Given the $EDB$ notion, an RDF graph $G$ can be represented by a set of $EDB$ i.e., $G=\cup_{e \in G}{B_e}$. By summarising all entity description blocks in $G$, we can get the intermediate summarisation result of $G$ i.e. $\cup_{e \in G}{P_e}$ . Given this intermediate result, we define a merge operation on EDPs which can further condense the result (c.f. Definition~\ref{edp:merge}).
\vspace{-1ex}
\begin{definition} 
\label{edp:merge}
(EDP Merge)  Given a set of EDPs $\mathcal{P}$, let $\mathcal{C}$ be the set of all class components in $\mathcal{P}$ and let $G_{\mathcal{P}}(c_i)$ be a subset of $\mathcal{P}$ whose elements share the same class components $c_i$. Then, \emph{merge} function can be defined as follows:
\scriptsize
\begin{equation}
Merge(\mathcal{P})=\{(c_i, \bigcup_{P_i \in G_{\mathcal{P}(c_i)}}{Attr(P_i)}, \bigcup_{P_i \in G_{\mathcal{P}(c_i)}}{Rel(P_i)}, \bigcup_{P_i \in G_{\mathcal{P}(c_i)}}{Rev(P_i)}) | c_i \in \mathcal{C} \}
\end{equation}
\normalsize
where
\begin{itemize}
\item $Attr(P_i)$ denotes the attribute component of $P_i$;
\item $Rel(P_i)$ denotes the relation component of $P_i$;
\item $Rev(P_i)$ denotes the reverse relation component of $P_i$.
\end{itemize}
\end{definition}

The rationale behind this merge operation is that entities of the same type(s) might be viewed as a set of homogeneous things. Given this idea, we can define an EDP function of an RDF graph as Definition~\ref{edp:func}.
\vspace{-1ex}
\begin{definition} 
\label{edp:func}
(EDP of RDF Graph) Given an RDF graph $G$, its EDP function is defined by the following equation.
\begin{equation}
EDP(G)=Merge(\bigcup_{e \in G}{P_e})
\end{equation}
\end{definition}

\noindent \textbf{EDP Graph} EDP function of an RDF graph results with a set of atomic graph patterns. Most data exploitation tasks can be decomposed into finding more complex graph patterns which are composed by these EDPs. To this end, it would be more beneficial to know how EDPs are connected to each other in the original RDF graph. Such information can be useful not only in decreasing search spaces (e.g., in query generation) but also for guiding the exploitation (e.g., browsing or linkage). With regards to this consideration, we introduce \emph{RDF data summarisation} as the notion of EDP graph (cf. Definition~\ref{edp:edpgraph}) for characterising the linking structures in the original RDF graph.
\vspace{-1ex}
\begin{definition} 
\label{edp:edpgraph} (EDP Graph) Given an RDF graph $G$, its EDP graph is defined as follows
\begin{equation}
\begin{split}
\mathcal{G}_{EDP}(G)= & 
\{<P_i,l,P_j>|\exists e_i \in E(P_i ), \exists e_j \in E(P_j ),<e_i,l,e_j> \in G, \\ 
& P_i \in EDP(G),P_j \in EDP(G) \}
\end{split}
\end{equation}
where $E(P_i)$ denotes the instances of EDP $P_i$. Specifically, if $P_i$ is not merged EDP, $E(P_i)$ is the set of entities whose EDP is $P_i$; if $P_i$  is a merged one, $E(P_i )=\cup_{P_k \in P}{E(P_k)}$, where $P$ is the set of EDPs from which $P_i$  is merged.
\end{definition}

\noindent \textbf{Annotated EDP Graph} EDP graphs are further annotated with statistic results. For each node $e$, it is annotated with a number which is the number of solutions to $Q(x) \leftarrow C_e(x)$. For each edge $l(C_e, C_f)$, there is a tuple of $(n_1, n_2)$, whose elements are the numbers of solutions to $Q_1(x) \leftarrow C_e(x), l(x,y), C_f(y)$ and $Q_2(y) \leftarrow C_e(x), l(x,y), C_f(y)$ respectively.