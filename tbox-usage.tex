Given an RDF graph, the summarisation is to generate a condensed description which can facilitate data exploitations. Our summarisation method applies a bottom-up strategy to summarize a semantic web dataset. Specifically, we propose an atomic pattern concept in which only one node is involved. Based on this concept, we summarise the given RDF dataset as a new graph which describes the relations between atomic graph patterns.

\vspace{-3ex}
\subsubsection{Entity Description Block}
In an RDF graph, we call its non-literal nodes as entities. For such an entity $e$ in an RDF graph $G$, we can get a data block for it by extracting triples in $G$ each of which has $e$ as its subject or object. We call such kind of data blocks as entity description blocks. Formally, each entity $e$ has an entity description block (EDB for short) as defined in Definition~\ref{def:edb}.

\begin{definition}
\label{def:edb}
 (Entity Description Block)
$\forall e \in G$, the description block of $e$ is defined as 
\begin{equation}
B_e= \{<e,p_i,o_i>|<e,p_i,o_i> \in G\} \cup \{<s_i,p_i,e>|<s_i,p_i,e> \in G\}
\end{equation}
where $s$ and $p$ are resources in G.
\end{definition}

\vspace{-5ex}
\subsubsection{Entity Description Pattern}
For an entity description block, it can be summarised by a notion of entity description pattern (Definition~\ref{def:edp}). EDP, the short name for entity description pattern, is the atomic graph pattern in our summarisation model. 

\begin{definition}
\label{def:edp} 
(Entity Description Pattern) Given an entity description block $B_e$, its description pattern is a tuple $P_e=(C_e,A_e,R_e,V_e)$, where
\begin{itemize}
\item $C_e=\{c_i |<e,rdf:type,c_i> \in G\}$   is called as the class component; 
\item $A_e=\{p_i |<e,p_i,l_i> \in G \text{ and $l_i$  is a literal}\}$  is called as the attribute component;
\item $R_e=\{r_i |<e,r_i,o_i> \in G \text{ and $o_i$  is a URI resource or blank node}\}$  is called as the relation component;
\item $V_e=\{v_i |<s_i,v_i,e> \in G\}$ is called as the reverse relation component.
\end{itemize}
\end{definition}

Given the $EDB$ notion, essentially, an RDF graph $G$ is a set of $EDB$ i.e. $G=\cup_{e \in G}{B_e}$. By summarizing all entity description blocks in $G$, we can get the intermediate summarization result of $G$ i.e. $\cup_{e \in G}{P_e}$ . Given this intermediate result, we define a merge operation on EDPs which can further condense the summarization. Definition 3 defines the merge operation on EDPs which share the same class component. 

\begin{definition} 
\label{edp:merge}
(EDP Merge)  Given a set of EDPs:$\{P_i\}_{i=1..n}$ whose elements have identical class component $C$, we can merge these EDPs into a representative EDP as follows:
\begin{equation}
Merge(\{P_i\}_{i=1..n})=(C, \bigcup_{i=1..n}{Attr(P_i)}, \bigcup_{i=1..n}{Rel(P_i)}, \bigcup_{i=1..n}{Rev(P_i)})
\end{equation}
where
\begin{itemize}
\item $Attr(P_i)$ denotes the attribute component of $P_i$;
\item $Rel(P_i)$ denotes the relation component of $P_i$;
\item $Rev(P_i)$ denotes the reverse relation component of $P_i$.
\end{itemize}
\end{definition}

The rationale behind this merge operation is that entities of the same type(s) might be viewed as a set of homogeneous things. Given this idea, we can define an EDP function of an RDF graph as Definition~\ref{edp:func}.

\begin{definition} 
\label{edp:func}
(EDP of RDF Graph) Given an RDF graph $G$, its EDP function is defined by the following equation.
\begin{equation}
EDP(G)=Merge(\bigcup_{e \in G}{P_e})
\end{equation}
\end{definition}

\vspace{-5ex}
\subsubsection{EDP Graph}
EDP function of an RDF graph results with a set of atomic graph patterns. Most data exploitation tasks can be decomposed into finding more complex graph patterns which are composed by these EDPs. To this end, it would be more beneficial to know how EDPs are connected to each other in the original RDF graph. Such information can be useful not only in decreasing search spaces (e.g., in query generation) but also for guiding the exploitation (e.g., browsing or linkage). With regards to this consideration, we introduce \emph{RDF data summarisation} as the notion of EDP graph (cf. Definition~\ref{edp:edpgraph}) for characterising the linking structures in the original RDF graph.

\begin{definition} 
\label{edp:edpgraph} (EDP Graph) Given an RDF graph $G$, its EDP graph is defined as follows
\begin{equation}
\begin{split}
\mathcal{G}_{EDP}(G)= & 
\{<P_i,l,P_j>|\exists e_i \in E(P_i ), \exists e_j \in E(P_j ),<e_i,l,e_j> \in G, \\ 
& P_i \in EDP(G),P_j \in EDP(G) \}
\end{split}
\end{equation}
where $E(P_i)$ denotes the set of entities conforms to the EDP $P_i$. If $P_i$ is not merged EDP, $E(P_i)$ is the set of entities from which $P_i$ can be generated; if $P_i$  is a merged one, $E(P_i )=\cup_{P_k \in P}{E(P_k)}$, $P$ is the set of EDPs from which $P_i$  is merged.
\end{definition}
