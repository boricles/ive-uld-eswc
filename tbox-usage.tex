\vspace{-2mm}
Given an RDF graph, the summarisation task is to generate a condensed description which can facilitate data exploitations. Different from existing ontology summarisation work, we put special emphasises on identifying a special type of basic graph patterns in RDF data, which is suitable for data exploitation.  The assumption of this special focus is that there exist such building blocks for revealing the constitution of an RDF dataset in a way which can not only help the understanding of the data but also is capable to guide RDF data exploitation. The rationale behind the assumption is that RDF data exploitation are usually based on graph patterns, e.g., SPARQL queries are based on BGP: basic  graph patterns.

%The main novelty of our summarisation approach is that it summarises an RDF graph by another much smaller graph structure based on atomic graph patterns. The linking structure in such summary graph can be utilised to significantly decrease search spaces in various data exploitation tasks e.g., query generation and query answering. Furthermore, statistic results of pattern instances are precomputed and attached to the summary, which can help both better understanding about the dataset and more efficient exploitation operations on it. 

Specifically, in this paper, we propose one definition of such building blocks, i.e., \emph{Entity Description Patterns} (EDPs for short). Given a resource $r$ in an RDF graph $G$, the description pattern of $r$ is $EDP(r, G)=\{C, A, P, R\}$, in which $C$ is the set of its classes, $A$ is a set of its data valued properties, $P$ is a set of its object properties, and $R$ is a set of $r$\rq{}s inverse properties.

While EDP is a way to summarise the descriptions of resources, the relations between resources can be characterised by connections between EDPs. Two EDP $E$ and $F$ is said to be linked by $p$ if and only if there exits $<r_e, p, r_f> \in G$ where $EDP(r_e, G)=E$ and $EDP(r_f, G)=F$.

Furthermore the statistics of EDP and their relations are useful information for guiding the data exploitation, e.g., the most cited papers can be interesting entities to the user. Hence, we provide statistical analysis result on EDPs and their relations. For each EDP $E$, it is annotated with a number which is the number of solutions to $Q(x) \leftarrow C_E(x)$. For each EDP relation $p(C_E, C_F)$, there is a tuple of $(n_1, n_2)$, whose elements are the numbers of solutions to $Q_1(x) \leftarrow C_E(x), p(x,y), C_F(y)$ and $Q_2(y) \leftarrow C_E(x), p(x,y), C_F(y)$ respectively.