% This is LLNCS.DEM the demonstration file of
% the LaTeX macro package from Springer-Verlag
% for Lecture Notes in Computer Science,
% version 2.3 for LaTeX2e
%
\documentclass{llncs}
%
\usepackage{makeidx}  % allows for indexgeneration
\usepackage{graphicx}
\usepackage{amssymb}
\usepackage{listings}
\usepackage{tabularx}
\usepackage{booktabs}
\usepackage{subscript}
\usepackage{amsmath}
\lstset{breaklines=true, basicstyle=\scriptsize\ttfamily}

%
\begin{document}
%
\frontmatter          % for the preliminaries
%
\pagestyle{headings}  % switches on printing of running heads
%\addtocmark{Hamiltonian Mechanics} % additional mark in the TOC
%
%
\mainmatter              % start of the contributions
%
\title{Exploiting Semantic Web Datasets: a Summarization Based Approach}

%
\titlerunning{Exploiting Semantic Web Datasets: a Summarization Based Approach}  % abbreviated title (for running head)
%                                     also used for the TOC unless
%                                     \toctitle is used
%
\author{Honghan Wu\inst{1} \and Boris Villazon-Terrazas\inst{2} \and Jeff Z. Pan\inst{1} \and Jose Manuel Gomez-Perez\inst{2}}
%
\authorrunning{Wu et al.}   % abbreviated author list (for running head)
%
%%%% list of authors for the TOC (use if author list has to be modified)
%\tocauthor{Panos Alexopoulos, Manolis Wallace}
%
\institute{Department of
Computing Science, University of Aberdeen, UK, \\ \email{\{honghan.wu,jeff.z.pan\}@abdn.ac.uk} \and
iSOCO, Intelligent Software Components S.A., Spain,\\
\email{bvillazon,jmgomez@isoco.com} }


\maketitle              % typeset the title of the contribution

%**************************************************************
%**************************************************************
%**************************************************************


\begin{abstract}
In the last years, we have witnessed vast increase of Linked Data datasets not only in the volume, but also in number of various domains and across different sectors. However, due to the nature and techniques used within Linked Data, it is non-trivial work for normal users to quickly understand what is within the datasets, and even for tech-users to efficiently exploit the datasets. In this paper, we propose a summarisation based approach to guide the exploitation of Linked Data. 
\end{abstract}

\vspace{-8mm}
\section{Introduction}\label{sec:Introduction}
So far, Linked Data principles and practices are being adopted by an increasing number of data providers, getting as result a global data space on the Web containing hundreds of LOD datasets \cite{Heath_Bizer_2011}. In this context it is important to promote the reuse and linkage of datasets, and to this end, it is necessary to know the structure of datasets. One step forward for knowing in depth the structure of a given dataset is to provide a summary of the dataset.

There are available works such as (1) \emph{LODStats}\footnote{\footnotesize \url{http://stats.lod2.eu/}} that provides the information related to a dataset, and (2) \emph{make-void} \footnote{\footnotesize \url{https://github.com/cygri/make-void}} that computes statistics about RDF files. However, LODStats is thought for the whole set of LOD datasets registered in The Data Hub \footnote{\footnotesize \url{http://thedatahub.com}}, and it is based on declarative descriptions of those datasets; and \emph{make-void} is thought for RDF files but not for RDF datasets.

In this paper we present fancy-name\footnote{\footnotesize \url{http://homepages.abdn.ac.uk/honghan.wu/pages/summ/}}, a simple tool to help users get a quick understanding of RDF dataset ...


\vspace{-8mm}
\section{Related Work}\label{sec:RelatedWork}
There are available works such as (1) \emph{LODStats}\footnote{\footnotesize \url{http://stats.lod2.eu/}} that provides the information related to a dataset, and (2) \emph{make-void} \footnote{\footnotesize \url{https://github.com/cygri/make-void}} that computes statistics about RDF files. However, LODStats is thought for the whole set of LOD datasets registered in The Data Hub \footnote{\footnotesize \url{http://thedatahub.com}}, and it is based on declarative descriptions of those datasets; and \emph{make-void} is thought for RDF files but not for RDF datasets.


\vspace{-4mm}
\section{RDF Summarisation: The EDP Graph}
\vspace{-2mm}
Given an RDF graph, the summarisation task is to generate a condensed description which can facilitate data exploitations. Different from existing ontology summarisation work, we put special emphasises on identifying a special type of basic graph patterns in RDF data, which is suitable for data exploitation.  The assumption of this special focus is that there exist such building blocks for revealing the constitution of an RDF dataset in a way which can not only help the understanding of the data but also is capable to guide RDF data exploitation. The rationale behind the assumption is that RDF data exploitation are usually based on graph patterns, e.g., SPARQL queries are based on BGP: basic  graph patterns.

%The main novelty of our summarisation approach is that it summarises an RDF graph by another much smaller graph structure based on atomic graph patterns. The linking structure in such summary graph can be utilised to significantly decrease search spaces in various data exploitation tasks e.g., query generation and query answering. Furthermore, statistic results of pattern instances are precomputed and attached to the summary, which can help both better understanding about the dataset and more efficient exploitation operations on it. 

Specifically, in this paper, we propose one definition of such building blocks, i.e., \emph{Entity Description Patterns} (EDPs for short). Given a resource $r$ in an RDF graph $G$, the description pattern of $r$ is $EDP(r, G)=\{C, A, P, R\}$, in which $C$ is the set of its classes, $A$ is a set of its data valued properties, $P$ is a set of its object properties, and $R$ is a set of $r$\rq{}s inverse properties.

While EDP is a way to summarise the descriptions of resources, the relations between resources can be characterised by connections between EDPs. Two EDP $E$ and $F$ is said to be linked by $p$ if and only if there exits $<r_e, p, r_f> \in G$ where $EDP(r_e, G)=E$ and $EDP(r_f, G)=F$.

Furthermore the statistics of EDP and their relations are useful information for guiding the data exploitation, e.g., the most cited papers can be interesting entities to the user. Hence, we provide statistical analysis result on EDPs and their relations. For each EDP $E$, it is annotated with a number which is the number of solutions to $Q(x) \leftarrow C_E(x)$. For each EDP relation $p(C_E, C_F)$, there is a tuple of $(n_1, n_2)$, whose elements are the numbers of solutions to $Q_1(x) \leftarrow C_E(x), p(x,y), C_F(y)$ and $Q_2(y) \leftarrow C_E(x), p(x,y), C_F(y)$ respectively.






\vspace{-5mm}
\section{Demos: The Summary Based Data Exploitations }
- Query Generation

- Reasoning  (DL-Lite materialisation of EDP graph)

- Data Set Enrichment


%\section{Evaluation}\label{sec:Evaluation}
%\subsection{Evaluation settings}

\subsection{Evaluation results}

\subsection{Evaluation discussion}

\vspace{-5mm}
\section{Conclusions and Future Work}\label{sec:Conclusions}
We described a dataset summarisation approach, which was shown to be useful for various data exploitation scenarios. The future work will focus on investigating the properties of the summary and in-depth studies in above scenarios.


%*****************************************************************************************************************



% ---- Bibliography ----
%
\vspace{-5mm}
\bibliography{bibliography}
\bibliographystyle{plain}



\clearpage
\addtocmark[2]{Author Index} % additional numbered TOC entry
\renewcommand{\indexname}{Author Index}
\printindex \clearpage
\addtocmark[2]{Subject Index} % additional numbered TOC entry
\markboth{Subject Index}{Subject Index}
\renewcommand{\indexname}{Subject Index}
%\input{subjidx.ind}
\end{document}
